\documentclass[dvipdfmx]{jsarticle}
%%%%%%
\usepackage{correction}
\newcommand{\bs}{\symbol{"5C}}

\begin{document}
\makecomment{theorem}{theorem 環境を使いましょう.}
\makecomment{displaystyle}{分数が潰れていて読み辛いので,\bs displaystyle を使ってディスプレイ数式にした方が良い(?)}
\makecomment[yshift=0pt]{comma}{カンマの後に空白を入れましょう}
\makecomment{usecommentOption}{\bs usecommentのオプション引数に負の値を指定するときは,予め(corlistingの外で)defしておく必要がある (何故?)}


\newcommand{\hoge}{-30pt}
\newcommand{\fuga}{-100pt}

\begin{corlisting}
\documentclass{jsarticle}
\usepackage{amsmath,amssymb}
\begin{document}

\section{調和関数}
\subsection{調和関数}

\begin{flushright}
$D\subset \mathbb{C} $ : 領域,連結,開集合

$z=x+iy$とする。
\end{flushright}

\underline{\textbf{Def 1.1.1}}@\usecomment{theorem}@

関数$u \colon D \to \mathbb{R}$が調和(harmonic)

$ \Leftrightarrow u \in C^2(D) $ かつ $ \triangle u = 0 $ on $ D $

\begin{itemize}
	\item $ \triangle = \frac{\partial ^2}{\partial x^2} + \frac{\partial ^2}{\partial y^2} = 4\frac{\partial ^2}{\partial z \partial \bar{z}}$@\usecomment{displaystyle}@

	$\frac{\partial}{\partial z} = \frac{1}{2} \frac{\partial}{\partial x} + \frac{1}{2i} \frac{\partial}{\partial y}$,
	$\frac{\partial}{\partial \bar{z}} = \frac{1}{2} \frac{\partial}{\partial x} + \frac{1}{-2i} \frac{\partial}{\partial y} $

	\item $ \triangle u $ はHessian $ \begin{pmatrix} u_{xx} & u_{xy} \\ u_{yx} & u_{yy} \end{pmatrix} $ のtrace

\end{itemize}

\underline{ex}@\usecomment{theorem}@

\begin{itemize}
	\item $u = x^2 - y^2$,@\usecomment[20pt]{comma}@$v = x^2-y^2$
  \item $u =\log \sqrt{x^2+y^2} = \log r$,@\usecomment[\hoge]{comma}\usecomment[\fuga]{usecommentOption}@$v= \tan ^{-1} \frac{y}{x} = \theta $

	($ \triangle r = \frac{1}{r} \neq 0 $)
\end{itemize}

\underline{\textbf{Prop 1.1.2}}

$ f \colon D \to \mathbb{C} $ 正則 に対し、

Re$f$,Im$f$は$D$上調和。

\underline{pf}

Cauchy-Riemannの等式より。 $\blacksquare$

局所的にはこの逆が成立。

$\linebreak$


(中略 by 若月)@\usecomment{comma}@


\end{document}
\end{corlisting}


\end{document}