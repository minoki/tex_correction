\documentclass[dvipdfmx]{jsarticle}
\usepackage{correction}
\newtcblistingExpand{corlisting}{\tcboptions,minted options={\mintedoptions}}
% エスケープ文字が @@ では不都合がある場合(提出されたソース中で使われている場合)は、上記 \newtcblistingExpand の部分をカスタマイズする。
% 例えば、エスケープ文字として || を使いたい場合は
% \newtcblistingExpand{corlisting}{\tcboptions,minted options={\mintedoptions,escapeinside=||,breakafter=)}}
% を、 ## を使いたい場合は
% \newtcblistingExpand{corlisting}{\tcboptions,minted options={\mintedoptions,escapeinside=\#\#,breakafter=)}}
% を使う。
% breakafterに = を指定するには breakafter={=} という風に波括弧で囲う。

\begin{document}

\begin{itemize}
  \item この添削ツールの使用に必要な準備
    \begin{itemize}
      \item \verb|pygments|のインストール: \verb|pip install pygments|とかで.(要\verb|python|)
      \item \verb|correction.sty|を\TeX が見つけられる場所に置く.
        例えば\verb|~/texmf/tex/|や,\verb|~/Library/texmf/tex/|など.
        実際の場所は\verb|kpsewhich -var-value=TEXMFHOME|を参考にすると良い.
    \end{itemize}
  \item 使い方
    \begin{itemize}
      \item \verb|python3 create.py /path/to/report.tex|とすると,
        \verb|/path/to/correction_report.tex|に添削用のファイルが生成される.
        (基本的にはコピペするだけだが,\verb|minted|のバグ(?)を回避するために,タブ文字をスペース2つに置換している)
      \item タイプセットは\verb|platex|だが,\verb|minted|を使うために\verb|-shell-escape|オプションが必要.
        また,正しい出力を得るには複数回のタイプセットが必要となる.
        適宜latexmkなりcluttexなりを使うと良いだろう.
      \item 受講生の書いたコードによっては,
        minted内のエスケープ文字や改行位置を変更したい場合もあるかもしれない.
        その場合は,この文章の\TeX ソース冒頭の\verb|\newtcblistingExpand|あたりのコメントを参考にすること.

    \end{itemize}
  \item 以下に色々とサンプルを載せておいたので,PDF (usage.pdf) と\TeX コード (usage.tex) を見比べつつ参考にしてください.
\end{itemize}

%%%%%%%%%%%%%%%%%%%%%%%%%%%%%%%%%%%%%%%%%%%%%%%%%%%%%%%%%%%%%%%%
\vspace{30pt}

\makecomment[yshift=10pt]{test}{コメントを書くテストです.複数行あったりしても平気.バックスラッシュは普通に\textbackslash textbackslash で.}
\makecomment[yshift=-30pt]{math}{数式中でも大丈夫.コメントが重なる場合はオプションでyshiftを指定すればずらせる}
\makecomment[yshift=-30pt]{item}{コメントの流用もできる.}
\makecomment[backgroundcolor=yellow!30,
             arrowcolor=green!50!blue,
             xshift=-10pt,
             yshift=-50pt,
             width=0.8\textwidth]
             {option}{オプションマシマシのコメントです.表示位置,幅,背景色,矢印の色を変更できます.}
\makecomment[arrowopacity=0,yshift=-70pt,backgroundcolor=blue!20]{opacity}{矢印を非表示(実装としては透明)にすることもできる}

\begin{corlisting}
\documentclass{jsarticle}

\begin{document}
\section{せくしょん}
ほげほげほげ $a+b=c$@\usecomment{test}@ 文章の間に矢印を挟んでも大丈夫.
$\int_0^1xdx@\usecomment{math}@=\frac{1}{2}$

\begin{itemize}
  \item@\usecomment{item}@ ほげ
  \item@\usecomment[-20pt]{item}@ ふが
  \item@\usecomment[-40pt]{item}@ ぴよ
\end{itemize}
\usecomment のオプション引数でy方向のshiftを指定できる.@\usecomment{opacity}@

@\ulstart@文章に下線@\ulend@を引いたり,@\boxstart@四角で囲ったり@\boxend@もできる.

@\boxstart[key=out,xshift=-10pt,yshift=1pt]@全体的に色々とオプションでいじれるようにした.
いちいち説明するのは@\boxstart[key=in]@面倒@\boxend[key=in]@なので,
この@\TeX@コード(usage.tex)に@\ulstart@オプションマシマシ@\ulend[color=blue]@のサンプルを載せておきます.@\boxend[key=out,color=green,linewidth=3,xshift=20pt,yshift=-20pt]@
色々と察してください.@\usecomment{option}@

\end{document}
\end{corlisting}
\end{document}